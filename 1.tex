\documentclass[12pt, a4paper]{article}
\usepackage[utf8]{inputenc}
\usepackage{xeCJK} % 支援繁體中文 (請使用 XeLaTeX 編譯)
\usepackage{amsmath}
\usepackage{amssymb}
\usepackage{listings}
\usepackage{xcolor}
\usepackage{geometry}
\usepackage{hyperref}

\geometry{margin=1in}
\setCJKmainfont{Noto Sans TC} % 請確保系統中有安裝繁體中文字型

\definecolor{codegreen}{rgb}{0,0.6,0}
\definecolor{codegray}{rgb}{0.5,0.5,0.5}
\definecolor{codepurple}{rgb}{0.58,0,0.82}
\definecolor{backcolour}{rgb}{0.95,0.95,0.92}

\lstset{
    backgroundcolor=\color{backcolour},   
    commentstyle=\color{codegreen},
    keywordstyle=\color{codepurple},
    numberstyle=\tiny\color{codegray},
    stringstyle=\color{codegreen},
    basicstyle=\ttfamily\footnotesize,
    breaklines=true,                 
    captionpos=b,                    
    keepspaces=true,                 
    numbers=left,                    
    showspaces=false,                
    showstringspaces=false,
    showtabs=false,                  
    tabsize=2,
    language=Python
}

\title{技術筆記:基於競爭失效風險與 PPO 的動態排程優化系統}
\author{工程學研究筆記 (由 Myakumyaku 整理)}
\date{\today}

\begin{document}

\maketitle

\section{系統概述 (System Overview)}
本研究旨在解決動態彈性作業車間排程問題 (DFJSP),並特別考慮了機器的物理衰退與隨機故障。核心框架結合了離散事件模擬 (DES) 環境與近接方策優化 (PPO) 演算法。

\section{馬可夫決策過程 (MDP) 建模}
系統被建模為一個五元組 $(\mathcal{S}, \mathcal{A}, P, R, \gamma)$。在程式碼中,環境由 \texttt{DFJSP\_EventEnv} 類別實現。

\subsection{狀態空間 ($\mathcal{S}$)}
狀態向量包含機器健康特徵與全域生產特徵。
\begin{itemize}
    \item \textbf{數學表達}:$s_t = [x_{m1}, \dots, x_{mN}, g_1, g_2, g_3]$
    \item \textbf{程式對應}:\texttt{exp.py} 中的 \texttt{\_get\_state(self)} 方法。
    \begin{lstlisting}[language=Python]
# 機器特徵包含:標準化狀態、可靠度、忙碌狀態、故障歷史
m_feats.extend([
    m.state / Config.K_STATES,
    m.get_reliability(),
    1.0 if m.status == 1 else 0.0,
    np.tanh(m.failure_count / 5.0)
])
    \end{lstlisting}
\end{itemize}

\section{物理感知模型 (Physics-informed Model)}
為了讓 Agent 感知機器的衰退情況,我們引入了韋伯分佈 (Weibull Distribution)。

\subsection{可靠度函數 (Reliability)}
可靠度 $R(t)$ 定義為機器在時間 $t$ 之前不發生故障的機率:
\begin{equation}
R(t) = \exp\left( - \left( \frac{a(t)}{\eta} \right)^\beta \right)
\end{equation}
其中 $a(t)$ 是累積機齡,$\eta = 500$ 是尺度參數,$\beta = 1.5$ 是形狀參數。
\begin{itemize}
    \item \textbf{程式對應}:\texttt{Machine.get_reliability(self)}。
    \begin{lstlisting}[language=Python]
def get_reliability(self):
    return math.exp(- (self.age_accum / Config.WEIBULL_ETA) ** Config.WEIBULL_BETA)
    \end{lstlisting}
\end{itemize}

\subsection{性能衰減與加工速度}
機器狀態 $S_k$ 會導致加工速度因子 $v(S_k)$ 下降:
\begin{equation}
v(S_k) = \max(0.5, 1.0 - S_k \times 0.05)
\end{equation}
\begin{itemize}
    \item \textbf{程式對應}:\texttt{Machine.get_processing_speed_factor(self)}。實際加工時間計算為 \texttt{actual\_time = base\_time / speed\_factor}。
\end{itemize}

\section{隨機故障與競爭失效 (Competing Risks)}
除了老化,機器還面臨波松過程 (Poisson Process) 的隨機故障風險。

\subsection{故障機率}
在加工時間 $t$ 內發生隨機故障的機率為:
\begin{equation}
P(\text{Breakdown}) = 1 - e^{-\lambda \cdot t}
\end{equation}
\begin{itemize}
    \item \textbf{程式對應}:\texttt{exp.py} 的 \texttt{step} 方法中模擬故障觸發。
    \begin{lstlisting}[language=Python]
breakdown_prob = 1 - math.exp(-Config.RANDOM_BREAKDOWN_RATE * actual_time)
if random.random() < breakdown_prob:
    # 觸發 BREAKDOWN 事件
    \end{lstlisting}
\end{itemize}

\section{動作遮罩 (Action Masking)}
為了縮減搜索空間,使用遮罩機制過濾非法動作。
\begin{itemize}
    \item \textbf{邏輯}:若機器處於完美狀態或無維修人力,則 PM 動作權重設為 $-\infty$。
    \item \textbf{程式對應}:\texttt{ActorCritic.act(self, state, mask)}。
    \begin{lstlisting}[language=Python]
inf_mask = (1.0 - mask) * -1e9
masked_logits = logits + inf_mask
dist = Categorical(logits=masked_logits)
    \end{lstlisting}
\end{itemize}

\section{獎勵函數塑形 (Reward Shaping)}
獎勵函數 $r_t$ 整合了延遲、成本與積壓懲罰:
\begin{equation}
r_t = -(w_1 C_{PM} + w_2 C_{CM} + w_3 \text{Tardiness} + w_4 \text{Backlog})
\end{equation}
\begin{itemize}
    \item \textbf{程式對應}:\texttt{exp.py} 中的 \texttt{Config} 權重設定。
    \begin{lstlisting}[language=Python]
W_TARDINESS = 1.0
W_PM_COST = 20.0
W_CM_COST = 100.0
W_BACKLOG = 0.5
    \end{lstlisting}
\end{itemize}

\section{PPO 優化 (PPO Optimization)}
PPO 透過裁剪 (Clipping) 策略更新幅度來確保穩定性:
\begin{equation}
L^{CLIP}(\theta) = \hat{\mathbb{E}}_t [ \min(r_t(\theta)\hat{A}_t, \text{clip}(r_t(\theta), 1-\epsilon, 1+\epsilon)\hat{A}_t) ]
\end{equation}
\begin{itemize}
    \item \textbf{程式對應}:\texttt{PPO.update(self)} 方法。
    \begin{lstlisting}[language=Python]
surr1 = ratios * advantages
surr2 = torch.clamp(ratios, 1-Config.EPS_CLIP, 1+Config.EPS_CLIP) * advantages
loss = -torch.min(surr1, surr2) + ...
    \end{lstlisting}
\end{itemize}

\section{結論}
根據實驗數據顯示,所提出的 PPO 框架在平衡維護成本與生產效率上優於 FIFO 與 SPT 等基準規則。這證明了將物理可靠度模型嵌入 DRL 狀態空間的有效性。

\end{document}